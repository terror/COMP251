\documentclass[10pt]{article}

\input{../../templates/set.tex}

\title{COMP 251: Algorithms and Data Structures - Proofs Assignment}
\author{Liam Scalzulli\\
\href{mailto:liam.scalzulli@mail.mcgill.ca}{liam.scalzulli@mail.mcgill.ca}}
\date{\today}

\begin{document}
\maketitle

\subsection*{Complexity Proof}

Claim: Inserting a node x into a red-black tree takes O(log n) time.

\subsubsection*{Presentation}

\subsubsection*{Summary}

\subsubsection*{Algorithm}

\subsubsection*{Code explanation}

\subsubsection*{Real world example}

\subsection*{Correctness Proof}

Claim: Red-Black Trees [CLRS 308]: A red-black tree with n internal nodes has height at most
2 log(n + 1).

\subsubsection*{Presentation}

\begin{proof}
  We start by showing that the subtree rooted at any node $x$ contains at least
  $2^{bh(x)} - 1$ internal nodes. We prove this claim by induction on the height
  of x. If the height of $x$ is 0, then x must be a leaf (T.nil), and the
  subtree rooted at $x$ indeed contains at least $2^{bh(x)} - 1$ = $2^0 - 1$ = 0
  internal nodes. For the inductive step, consider a node $x$ that has positive
  height and is an internal node with two children. Each child has a
  black-height of either $bh(x)$ or $bhx - 1$, depending on whether its color is
  red or black, respectively. Since the height of a child of $x$ is less than
  the height of $x$ itself, we can apply the inductive hypothesis to conclude
  that each child has at least $2^{bh(x) - 1} - 1$ internal nodes. Thus, the
  subtree rooted at $x$ contains at least ($2^{bh(x) - 1} - 1$) + ($2^{bh(x) - 1} -
  1$) + 1 = $2^{bh(x)} - 1$ internal nodes, which proves the claim.
\end{proof}

\noindent
Source: CLRS

\subsubsection*{Summary}

\subsubsection*{Algorithm}

\subsubsection*{Code explanation}

\subsubsection*{Real world example}

\end{document}
