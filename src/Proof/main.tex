\documentclass[10pt]{article}


\usepackage {
  amsmath,
  amssymb,
  amsthm,
  array,
  graphicx,
  multicol,
  longdivision
}

\usepackage[colorlinks=true]{hyperref}
\usepackage[dvipsnames]{xcolor}
\usepackage[english]{babel}
\usepackage[margin=1in]{geometry}
\usepackage[utf8]{inputenc}
\usepackage{color}
\usepackage{circuitikz}

\hypersetup {
  citecolor = ForestGreen,
  filecolor = Plum,
  linkcolor = NavyBlue,
  urlcolor = RubineRed
}

\newcommand{\C}{\mathbb{C}}
\newcommand{\F}{\mathbb{F}}
\newcommand{\Q}{\mathbb{Q}}
\newcommand{\Z}{\mathbb{Z}}
\newcommand{\R}{\mathbb{R}}
\newcommand{\N}{\mathbb{N}}
\newcommand{\CO}{\mathcal{O}}
\newcommand{\CC}{\mathcal{C}}
\newcommand{\CU}{\mathcal{U}}
\newcommand{\spacing}{\vspace*{1\baselineskip}}


\title{COMP
251:
Algorithms
and
Data
Structures
-
Proofs
Assignment}
\author{Liam
Scalzulli\\
\href{mailto:liam.scalzulli@mail.mcgill.ca}{liam.scalzulli@mail.mcgill.ca}}
\date{\today}


\begin{document}
  \maketitle


  \subsection*{Complexity Proof}


  Claim: Inserting a node $x$ into a red-black tree takes $O(log n)$ time.

  \subsubsection*{Presentation}


  \begin{proof}
    To prove that inserting a node $x$ into a red-black tree takes $O(log n)$ time,
    we need to show that the number of operations performed during the insertion
    operation is proportional to the height of the tree, which is $O(logn)$ as shown
    in the correctness proof.

    \spacing
    \noindent
    The insertion operation in a red-black tree can be divided into three main steps:

    \begin{enumerate}
      \item \textbf{Binary search tree insertion}: This step involves finding the
        correct position for the new node $x$ in the tree based on its key value,
        and inserting it as a regular binary search tree node.

      \item \textbf{Coloring the new node red}: The new node is colored red to preserve
        the red-black tree properties.

      \item \textbf{Fixing the red-black tree properties}: If the insertion of the
        new node causes a violation of the red-black tree properties, then one
        or more rotations and/or color flips are performed to restore the
        properties.
    \end{enumerate}

    \noindent
    Let $h$ be the height of the tree before the insertion operation, and let $h'$
    be the height of the tree after the operation. Since step 1 involves a
    standard binary search tree insertion, the number of comparisons performed during
    this step is proportional to the height of the tree, i.e., $O(h)$. Therefore,
    the worst-case time complexity of step 1 is $O(h)$, which is $O(logn)$ in the
    average case for a balanced tree.

    \spacing
    \noindent
    Step 2 involves a constant number of operations, namely setting the color of
    the new node to red. Therefore, the time complexity of this step is $O(1)$.

    \spacing
    \noindent
    Step 3 involves fixing any violations of the red-black tree properties that
    may have occurred due to the insertion of the new node. The number of operations
    required to fix these violations is proportional to the height of the subtree
    affected by the violation. Since the subtree height is at most h+1, the worst-case
    time complexity of step 3 is $O(h+1)$, which is $O(log n)$ in the average
    case for a balanced tree.

    \spacing
    \noindent
    Therefore, the total worst-case time complexity of inserting a node into a red-black
    tree is $O(log n)$. This is because step 1 takes $O(log n)$, step 2 takes $O(
    1)$, and step 3 takes $O(log n)$. Therefore, the total number of operations
    is proportional to the height of the tree, which is $O(log n)$.
  \end{proof}

  \noindent
  Source: Slides

  \subsubsection*{Summary}


  The proof above shows that inserting a node $x$ into a red-black tree takes $O(
  logn)$ time. The insertion involves three steps: binary search tree insertion,
  coloring the new node red, and fixing any violations of the red-black tree properties.
  The time complexity of step 1 is $O(h)$, which is $O(logn)$ in the average case
  for a balanced tree (as shown in the correctness proof). Step 2 takes $O(1)$
  and step 3 takes $O(h+1)$, which is also O(logn) in the average case. Therefore,
  the total worst-case time complexity of inserting a node into a red-black tree
  is $O(logn)$.

  \subsubsection*{Algorithm}


  \subsubsection*{Code explanation}


  \subsubsection*{Real world example}


  \subsection*{Correctness Proof}


  Claim: Red-Black Trees [CLRS 308]: A red-black tree with n internal nodes has
  height at most 2 log(n + 1).

  \subsubsection*{Presentation}


  \begin{proof}
    We start by showing that the subtree rooted at any node $x$ contains at least
    $2^{bh(x)}- 1$ internal nodes. We prove this claim by induction on the
    height of x. If the height of $x$ is 0, then x must be a leaf (T.nil), and the
    subtree rooted at $x$ indeed contains at least $2^{bh(x)}- 1$ = $2^{0}- 1$ =
    0 internal nodes. For the inductive step, consider a node $x$ that has positive
    height and is an internal node with two children. Each child has a black-height
    of either $bh(x)$ or $bhx - 1$, depending on whether its color is red or black,
    respectively. Since the height of a child of $x$ is less than the height of $x$
    itself, we can apply the inductive hypothesis to conclude that each child
    has at least $2^{bh(x) - 1}- 1$ internal nodes. Thus, the subtree rooted at
    $x$ contains at least ($2^{bh(x) - 1}- 1$) + ($2^{bh(x) - 1}- 1$) + 1 =
    $2^{bh(x)}- 1$ internal nodes, which proves the claim.
  \end{proof}

  \noindent
  Source: CLRS

  \subsubsection*{Summary}


  The proof above uses induction to show that the subtree rooted at any node x
  in a red-black tree contains at least $2^{bh(x)}- 1$ internal nodes, where $bh(
  x)$ is the black-height of $x$. The base case is when $x$ is a leaf, and the inductive
  step applies the inductive hypothesis to the two children of x and uses some arithmetic
  to derive the desired lower bound.

  \subsubsection*{Algorithm}


  \subsubsection*{Code explanation}


  \subsubsection*{Real world example}
\end{document}